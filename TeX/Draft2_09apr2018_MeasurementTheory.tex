\PassOptionsToPackage{unicode=true}{hyperref} % options for packages loaded elsewhere
\PassOptionsToPackage{hyphens}{url}
%
\documentclass[]{article}
\usepackage{lmodern}
\usepackage{amssymb,amsmath}
\usepackage{ifxetex,ifluatex}
\usepackage{fixltx2e} % provides \textsubscript
\ifnum 0\ifxetex 1\fi\ifluatex 1\fi=0 % if pdftex
  \usepackage[T1]{fontenc}
  \usepackage[utf8]{inputenc}
  \usepackage{textcomp} % provides euro and other symbols
\else % if luatex or xelatex
  \usepackage{unicode-math}
  \defaultfontfeatures{Ligatures=TeX,Scale=MatchLowercase}
\fi
% use upquote if available, for straight quotes in verbatim environments
\IfFileExists{upquote.sty}{\usepackage{upquote}}{}
% use microtype if available
\IfFileExists{microtype.sty}{%
\usepackage[]{microtype}
\UseMicrotypeSet[protrusion]{basicmath} % disable protrusion for tt fonts
}{}
\IfFileExists{parskip.sty}{%
\usepackage{parskip}
}{% else
\setlength{\parindent}{0pt}
\setlength{\parskip}{6pt plus 2pt minus 1pt}
}
\usepackage{hyperref}
\hypersetup{
            pdfborder={0 0 0},
            breaklinks=true}
\urlstyle{same}  % don't use monospace font for urls
\usepackage{longtable,booktabs}
% Fix footnotes in tables (requires footnote package)
\IfFileExists{footnote.sty}{\usepackage{footnote}\makesavenoteenv{longtable}}{}
\usepackage{graphicx,grffile}
\makeatletter
\def\maxwidth{\ifdim\Gin@nat@width>\linewidth\linewidth\else\Gin@nat@width\fi}
\def\maxheight{\ifdim\Gin@nat@height>\textheight\textheight\else\Gin@nat@height\fi}
\makeatother
% Scale images if necessary, so that they will not overflow the page
% margins by default, and it is still possible to overwrite the defaults
% using explicit options in \includegraphics[width, height, ...]{}
\setkeys{Gin}{width=\maxwidth,height=\maxheight,keepaspectratio}
\setlength{\emergencystretch}{3em}  % prevent overfull lines
\providecommand{\tightlist}{%
  \setlength{\itemsep}{0pt}\setlength{\parskip}{0pt}}
\setcounter{secnumdepth}{0}
% Redefines (sub)paragraphs to behave more like sections
\ifx\paragraph\undefined\else
\let\oldparagraph\paragraph
\renewcommand{\paragraph}[1]{\oldparagraph{#1}\mbox{}}
\fi
\ifx\subparagraph\undefined\else
\let\oldsubparagraph\subparagraph
\renewcommand{\subparagraph}[1]{\oldsubparagraph{#1}\mbox{}}
\fi

% set default figure placement to htbp
\makeatletter
\def\fps@figure{htbp}
\makeatother

\usepackage{color}
\usepackage{soulutf8}
\usepackage{color}
\usepackage{soulutf8}

\date{}

\begin{document}

\hypertarget{header-n0}{%
\subsection{Experimental Theory and Design}\label{header-n0}}

\hypertarget{header-n2}{%
\subsubsection{Introduction}\label{header-n2}}

There are several steady-state and transient techniques available for
the measurement of thermal properties. Although not as accurate as
steady-state methods such as the Guarded Hot Plate apparatus, transient
techniques offer important advantages over steady-state techniques.
These include simultaneous measurement of thermal conductivity and
thermal diffusivity, significantly shorter duration of experiments and
capability of studying moisture effects on thermal properties. Moreover,
transient measurement setups are generally simpler and could be built at
a lower cost compared to the steady-state counterparts.

There are several transient measurement methods {[}1{]}, distinguished
by the type (wire, strip or plane) and thermal profile (pulse or step)
of heat source and the point of temperature measurement (at source or
away from the source). One of the first techniques was implemented by
Gustafsson {[}2{]} in 1991 using a Transient Plane Source element (now
known as the "Gustafsson Probe") which works as both the heat source and
the sensor. Transient method involving step-wise heating was discussed
by Kubicar and Bohac {[}3{]}. Lei et al. {[}4{]} used the transient
plane source method with stepwise heating to measure thermal properties
of fabrics which are fibrous in nature. Recently, Malinarič {[}5{]}
carried out a detailed analysis of Step-wise transient method with disk
shaped plane heat source.

\hypertarget{header-n7}{%
\subsubsection{Measurement Theory}\label{header-n7}}

In the current work, Transient Plane Source (TPS) with a step heat flux
is used for the measurement of thermal conductivity and thermal
diffusivity. The temperature profile is captured by a sensor positioned
at a fixed distance from the heat source.

\hypertarget{header-n10}{%
\paragraph{Semi-infinite Model}\label{header-n10}}

The experimental setup is designed to replicate the 1-D semi-infinite
transient model illustrated in Figure 1 during the duration of the
experiment. The analytical solution for the model is given by,

\[T(x,t) = \frac{2q}{k} \left[ \sqrt{\frac{\alpha t}{\pi}}\exp{\frac{-x^2}{4 \alpha t}} - \frac{x}{2}erfc(\frac{x}{2 \sqrt{\alpha t}})\right]\]

where q is the heat flux and x is the distance between the sensor and
the heat source. The sample can be approximated as a semi-infinite solid
to a certain degree of accuracy if the following criterion is satisfied:

\[Fourier Number  \phantom{1} Fo = \frac{\alpha t }{L^2 } \leq 0.2\]

where, t is the measurement time and L is equal to half the thickness of
the sample. An approximate maximum value of thermal diffusivity,
\(\alpha_{max}\) (obtained from literature or other sources) for the
material of the sample or similar material shall be used in the above
condition. By setting a practical value of sample thickness, the maximum
duration of the experiment is obtained as follows.

\[t_{max} = \frac{0.2L^2}{\alpha_{max}}\]

\hl{With the thickness of the sample L fixed, the width (for a square
shaped sample) is fixed to be about 2.4 times the thickness as suggested
by Lei et al. {[}6{]}.}

The temperature profile obtained from the experiment is curve-fit to the
analytical solution to get k and \(\alpha\) using non-linear parameter
estimation procedure.

\hypertarget{header-n24}{%
\paragraph{Sensitivity Coefficient Analysis}\label{header-n24}}

 The selection of the time window to be used for the curve fitting
procedure depends on the sensitivity of the parameters to the change in
temperature as well as the linear dependence of the parameters to each
other. Sensitivity coefficient analysis is to be carried out for this
purpose.

 Sensitivity Coefficient is defined as

\[\beta_{p}(t) = p \frac{\partial T}{\partial p}\]

where p is an unknown parameter. If the sensitivity coefficients are low
or linearly dependent on each other, the curve fit may not work
properly. Here, the unknown parameters are k and \(\alpha\) for which
the corresponding sensitivity coefficient expressions are given by,

\[\beta_{k} =\frac{-2q}{k} \left[ \sqrt{\frac{\alpha t}{\pi}}\exp{\frac{-x^2}{4 \alpha t}} - \frac{x}{2}erfc(\frac{x}{2 \sqrt{\alpha t}})\right]\]

\[\beta_{\alpha} = \frac{q}{k}\sqrt{\frac{\alpha t}{\pi}}\exp{\frac{-x^2}{4 \alpha t}}\]

The plot of \(\beta_{k}\) and \(\beta_{\alpha}\) against time or Fo
indicates the initial period of time during which the properties are not
sensitive to the change in temperature because of which their values
cannot be estimated accurately. Since the less diffusive materials (low
\(\alpha\)) tend to take longer to respond to the applied heat flux, we
shall consider an approximate value of \(\alpha_{min}\) of the sample
under study. \hl{The linear dependence between the parameters can be}

To apply this analysis for different materials, we make use of the
normalized forms of the sensitivity coefficients defined as,

\[\beta_p^{'} =\left|\frac{\beta_p - \beta_{p,0}}{\beta_{p,max} - \beta_{p,0}}\right|\]

 where \(\beta_{p,0}\) is the sensitivity coefficient value at the start
of the measurement (at time \(t = 0\)) and and \(\beta_{p,max}\) is the
same at the end of the measurement (at time \(t=t_m\) )

\hypertarget{header-n41}{%
\paragraph{Standard and Difference Analysis}\label{header-n41}}

\hypertarget{header-n44}{%
\paragraph{Uncertainty Analysis}\label{header-n44}}

 For calculating the standard deviation of the estimated parameters, the
uncertainty analysis method described in Beck and Arnold {[}7{]} and
implemented in {[}5{]} is used here.

 For the set of parameters \(p_j\), the standard deviation of the
least-square estimate is obtained as shown below:

\[\sigma^2(p_j) = \{(X^T.X)^{-1}\}_{jj}\sigma^2\]

where \(\sigma\) is the standard deviation of the measurement system and
X is the sensitivity matrix defined as,

\[\{X\}_{jj} = \frac{\partial T(t_i, \bar{p})}{\partial p_j}\]

\hypertarget{header-n53}{%
\subsubsection{Experimental Setup}\label{header-n53}}

 The heating element consisting of a Nickel foil (25 micron thickness),
having a square shape and spiral structure with thin layers of kapton on
either sides acts as the plane heat source. It is heated by a DC power
supply under constant current setting. Two identical samples are placed
sandwiching the heat source and held under force through a pressing
mechanism to ensure good contact between the sample and heating element.
Heat conducting pastes are also applied on the element to minimize
contact resistance existing because of the presence of air gaps.
Temperature is measured using K-type thermocouples which are positioned
by appropriate methods. For the case of solid samples, a hole is drilled
in the radial direction in such a way that the thermocouple is held
tightly in the desired position. In the case of fibrous materials, the
positioning is more difficult and hence a support made of a similar
material to the sample is used to position the thermocouple. A thin
wooden stick with a groove to place the thermocouple inside is crafted
for the straw samples. Once again, heat pastes are used to reduce the
contact resistance existing between the tip of the sensor and the
contact area in the sample. The entire setup is placed inside a climate
chamber in order to study the effects of temperature and relative
humidity on the thermal properties of the sample. Two ... sensors
interfaced with Arduino controller is used to monitor the chamber
humidity and temperature. Additionally, two Thermocouples are also
monitor the chamber temperature. The temperature data is logged in a
desktop computer through the Agilent data acquisition system.
Calibration of the thermocouples and the relative humidity sensors were
carried out before the start of the experiment.\\

\hypertarget{header-n56}{%
\subsection{Results and Discussion}\label{header-n56}}

 Polymethyl Methacrylate (PMMA) and Extruded Polystyrene (XPS) are the
reference samples considered for validating the experimental method
while the material under study is rice paddy straw. For all the cases,
the sensors are placed 1 cm away from the source on either sides. The
sample size is determined from the semi-infinite condition mentioned
above. The time window (\(t_0\) to \(t_m\)) for curve-fitting is chosen
by performing the analyses discussed above for each sample. The results
of the analyses are presented in Figure 2 and 3. The final experimental
design values are listed in Table 1.

\begin{longtable}[]{@{}l|c|c|c@{}}
\toprule
& PMMA & XPS & Straw\tabularnewline
\midrule
\endhead
\(\rho\) (measured) {[}\(kg/m^3\){]} & \(1172.97\) & \(39.51\) &
\(70-130\)\tabularnewline
\(k_{max} [W/mK]\) & \(0.21\){[}Rides{]} & \(0.042\){[}Dubois{]} &
\(0.12\) {[}Chaussinand{]}\tabularnewline
\(C_{min}\){[}\(kJ/kgK\){]} & - & \(1280\){[}Al-Ajlan{]} &
\(1338\){[}Chaussinand{]}\tabularnewline
\(\alpha_{max}\) {[}\(m^2/s\){]} & \(1.15 \times 10^{-7}\){[}Rides{]} &
\(8.31 \times 10^{-7}\) & \(1.28 \times 10^{-6}\)\tabularnewline
L (set) {[}\(m\){]} & \(0.03\) & \(0.081\) & \(0.1\)\tabularnewline
\(\approx t_{max}\){[}\(s\){]} & \(390\) & \(395\) &
\(335\)\tabularnewline
\(k_{min} [W/mK]\) & \(0.18\){[}Rides{]} & \(0.032\){[}Dubois{]} &
\(0.052\){[}Chaussinand{]}\tabularnewline
\(C_{max}\){[}\(kJ/kgK\){]} & - & \(1280\) &
\(2000\){[}Chaussinand{]}\tabularnewline
\(\alpha_{min}\) {[}\(m^2/s\){]} & \(9.0 \times 10^{-8}\){[}Rides{]} &
\(6.33 \times 10^{-7}\) & \(2.0 \times 10^{-7}\)\tabularnewline
\(t_0\) {[}\(s\){]} & \(100\) & \(50\) & \(80\)\tabularnewline
& & &\tabularnewline
\bottomrule
\end{longtable}

\includegraphics{H:/Dropbox/Research/Experiment/Results/TransientAnalysis.png}

\begin{figure}
\centering
\includegraphics{H:/Dropbox/Research/Experiment/Results/PMMA_std_diff_NOerror__GREYSCALE.png}
\caption{}
\end{figure}

\begin{figure}
\centering
\includegraphics{H:/Dropbox/Research/Experiment/Results/XPS_std_diff_NOerror_GREYSCALE.png}
\caption{}
\end{figure}

\begin{figure}
\centering
\includegraphics{H:/Dropbox/Research/Experiment/Results/Straw_std_diff_NOerror.csv.png}
\caption{}
\end{figure}

\begin{figure}
\centering
\includegraphics{H:/Dropbox/Research/Experiment/Results/PMMA_std_witherror_GREYSCALE.png}
\caption{}
\end{figure}

\begin{figure}
\centering
\includegraphics{H:/Dropbox/Research/Experiment/Results/XPS_std_witherror_GREYSCALE.png}
\caption{}
\end{figure}

\begin{figure}
\centering
\includegraphics{H:/Dropbox/Research/Experiment/Results/Straw_std_witherror_GREYSCALE.png}
\caption{}
\end{figure}

\begin{figure}
\centering
\includegraphics{H:/Dropbox/Research/Experiment/Results/PMMA_diff_witherror_GREYSCALE.png}
\caption{}
\end{figure}

\begin{figure}
\centering
\includegraphics{H:/Dropbox/Research/Experiment/Results/XPS_diff_witherror_GREYSCALE.png}
\caption{}
\end{figure}

\begin{figure}
\centering
\includegraphics{H:/Dropbox/Research/Experiment/Results/Straw_diff_witherror_GREYSCALE.png}
\caption{}
\end{figure}

\hypertarget{header-n152}{%
\subsection{References}\label{header-n152}}

{[}1{]} Tye, R. P., and N. Lockmuller. "The development of a standard
for contact transient methods of measurement of thermophysical
properties." \emph{International journal of thermophysics} 26.6 (2005):
1917-1938.

{[}2{]} Gustafsson, Silas E. "Transient plane source techniques for
thermal conductivity and thermal diffusivity measurements of solid
materials." \emph{Review of scientific instruments} 62.3 (1991):
797-804.

{[}3{]} Ludovit Kubicar and Bohác, Vlastimil. "A step-wise method for
measuring thermophysical parameters of materials." \emph{Measurement
Science and Technology} 11.3 (2000): 252.

{[}4{]} Lei, Zuo, Sukang Zhu, and Ning Pan. "Transient methods of
thermal properties measurement on fibrous materials." \emph{Journal of
Heat Transfer} 132.3 (2010): 032601.

{[}5{]} Malinarič, Svetozár. "Step-wise transient method."
\emph{Measurement Science and Technology} 27.3 (2016): 035601.

{[}6{]} Lei, Zuo, Sukang Zhu, and Ning Pan. "Determination of sample
size for step-wise transient thermal tests." \emph{Polymer testing} 28.3
(2009): 307-314.

{[}7{]} Beck, James Vere, and Kenneth J. Arnold. \emph{Parameter
estimation in engineering and science}. James Beck, 1977.

\end{document}
