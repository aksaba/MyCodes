\PassOptionsToPackage{unicode=true}{hyperref} % options for packages loaded elsewhere
\PassOptionsToPackage{hyphens}{url}
%
\documentclass[]{article}
\usepackage{lmodern}
\usepackage{amssymb,amsmath}
\usepackage{ifxetex,ifluatex}
\usepackage{fixltx2e} % provides \textsubscript
\ifnum 0\ifxetex 1\fi\ifluatex 1\fi=0 % if pdftex
  \usepackage[T1]{fontenc}
  \usepackage[utf8]{inputenc}
  \usepackage{textcomp} % provides euro and other symbols
\else % if luatex or xelatex
  \usepackage{unicode-math}
  \defaultfontfeatures{Ligatures=TeX,Scale=MatchLowercase}
\fi
% use upquote if available, for straight quotes in verbatim environments
\IfFileExists{upquote.sty}{\usepackage{upquote}}{}
% use microtype if available
\IfFileExists{microtype.sty}{%
\usepackage[]{microtype}
\UseMicrotypeSet[protrusion]{basicmath} % disable protrusion for tt fonts
}{}
\IfFileExists{parskip.sty}{%
\usepackage{parskip}
}{% else
\setlength{\parindent}{0pt}
\setlength{\parskip}{6pt plus 2pt minus 1pt}
}
\usepackage{hyperref}
\hypersetup{
            pdfborder={0 0 0},
            breaklinks=true}
\urlstyle{same}  % don't use monospace font for urls
\setlength{\emergencystretch}{3em}  % prevent overfull lines
\providecommand{\tightlist}{%
  \setlength{\itemsep}{0pt}\setlength{\parskip}{0pt}}
\setcounter{secnumdepth}{0}
% Redefines (sub)paragraphs to behave more like sections
\ifx\paragraph\undefined\else
\let\oldparagraph\paragraph
\renewcommand{\paragraph}[1]{\oldparagraph{#1}\mbox{}}
\fi
\ifx\subparagraph\undefined\else
\let\oldsubparagraph\subparagraph
\renewcommand{\subparagraph}[1]{\oldsubparagraph{#1}\mbox{}}
\fi

% set default figure placement to htbp
\makeatletter
\def\fps@figure{htbp}
\makeatother


\date{}

\begin{document}

\hypertarget{header-n0}{%
\section{Thermal Comfort in Residential Buildings with Passive Cooling
techniques utilizing Straw based Insulation in the Building
Envelope}\label{header-n0}}

\hypertarget{header-n2}{%
\subsection{INTRODUCTION}\label{header-n2}}

\hypertarget{header-n3}{%
\subsection{LITERATURE REVIEW}\label{header-n3}}

\hypertarget{header-n4}{%
\subsubsection{1. Passive Cooling of buildings}\label{header-n4}}

\hypertarget{header-n5}{%
\subsubsection{2. Agriculture Residue based
Insulations}\label{header-n5}}

\hypertarget{header-n6}{%
\subsubsection{3. Thermo-physical and Hygroscopic Properties of
Straw}\label{header-n6}}

\hypertarget{header-n7}{%
\paragraph{Thermal Transport Properties}\label{header-n7}}

\hypertarget{header-n8}{%
\subparagraph{Thermal Conductivity}\label{header-n8}}

 Thermal conductivity, denoted by k or \(\lambda\), is a fundamental
thermal transport property of a material appearing in the formulation of
Fourier's law:

\[Q = - kA\frac{dT}{dx}\]

Thermal conductivity for insulations used as construction material for
buildings indicates the capability of the material to prevent heat from
entering/leaving the indoor space which in turn affects the thermal
comfort conditions inside. There exists several methods for evaluation
of thermal conductivity for different types of materials which shall be
briefly touched upon later. For the case of straw, which can be
considered as a fibrous porous material in bulk quantities, an effective
thermal conductivity can be measured through these methods. Straw being
organic and hygroscopic in nature, its effective thermal conductivity is
influenced by an array of factors including type of crop, local climatic
conditions, composition, bulk density, moisture content, etc. In the
following paragraphs, attempts by researchers to determine thermal
conductivity of straw and its influencing factors is reviewed and direct
measurement of thermal conductivity is dealt with first.

 Lebed and Augaitis (2017) measured the thermal conductivity of
triticale straw samples with varying densities in the range of 80
kg/m\(^{3}\) and 190 kg/m\( ^{3}\). Guarded hot plate apparatus was used
to measure the thermal conductivity and the tests were conducted at a
mean sample temperature of 10 \(^{\circ}\)C and ambient conditions of
23\(\pm\)2 \(^{\circ}\)C; 50\(\pm\)5 \% RH. Based on the experimental
data, they had arrived at an equation relating thermal conductivity of
straw and its density as shown below:

\[\lambda_{10^{\circ}C} = -0.00155 + 0.000357.\rho  + \frac{3.381}{\rho}\]

where, \(\rho\) is the density of straw infill. The researchers observed
an increasing trend in thermal conductivity values when density in
increased beyond 120 kg/m\(^{3}\).

 Costes et al. (2017) begin with a good review of previous measurements
of thermal conductivity of straw including its variation with density
and orientation. They argue that samples prepared from straw bales may
not be representative of the actual material due to changes in
orientation and density introduced during the sample preparation process
and consequently influencing its thermal conductivity value. Therefore,
the authors have designed and developed a Guarded hot plate apparatus
capable of measuring thermal conductivity of samples of up to 50 cm
thick. The authors prepared 13 straw bale samples of wide varying
densities (68.1 kg\(/m^{3}\)- 122.7 kg/m\(^{3}\)) but small variations
in thickness (0.3 m - 0.495 m) and measurements were carried out on each
sample five times . Regression of the measurement data resulted in the
following correlation:

\[\lambda = 4.81 \times 10^{-2} + 2.9 \times 10^{-4} d   -1.13 \times 10^{-2} t\]

where d is the density in kg/m\(^{3}\) and t is the thickness in m.

 F. D' Alessandro et al. (2017) also used guarded hot plate apparatus to
measure the thermal conductivity of straw samples of density 80
kg/m\(^{3}\). Two tests conducted resulted in thermal conductivity
values of 0.051 W/mK and 0.053 W/mK. The author notes that the straw
bales were in flat position meaning the heat flow perpendicular to the
straw fiber orientation.

 Douzane et al. (2016) reported thermal conductivity values of straw of
density 80 kg/m\(^{3}\) considering different orientations. Guarded hot
plate techniques was used to to conduct the measurement. At 10°C and dry
conidions, thermal conductivity values of 0.0723 \(\pm\) 0.0014 W/mK was
reported for parallel orientation while 0.0510 \(\pm\) 0.0010 W/mK was
reported for perpendicular orientation. Also measurements performed at
various sample mean temperatures revealed a direct linear relation
between thermal conductivity and sample mean temperature. \\

 Conti et al. (2016) attempted in building a low-cost system with
established standards acting as guidelines to measure thermal
conductivity of straw bale with reasonable accuracy. Local agricultural
practices and climatic conditions leading to wide variations in the
thermal behaviour of straw bale is put forward as the reason behind
devising such a system. Thermal conductivities of two straw bale samples
of approximate densities of 65 and 85 kg/m\(^{3}\) were measured to be
0.062 W/mK and 0.070 W/mK respectively.

 Walker et al. (2016) reviewed the previous works in which thermal
conductivity of straw had been measured. It included measurement of
thermal conductivity of 120 kg/m\(^{3}\) dense straw bale by Shea et al.
(2013) who had reported a value of 0.064 W/mK. With regards to
orientation of straw sample, the authors cited the results published by
the German national organisation of straw bale building (FASBA, 2009)
which were equal to 0.067 W/mK for straw oriented parallel to heat flow
and 0.043 W/mK for straw oriented perpendicular to heat flow. Further,
considering straw moisture content as a parameter, the authors reported
the work carried out by a French company CEBTP in 2004 in which a modest
increase from 0.064 to 0.072 W/mK was observed when straw relative
humidity was increased from 0\% to 90\%.\\

 Samuel et al. (2015) used a previously designed guarded hot plate
apparatus by Dubois and Lebeau (2013) to measure their own sample of
straw. They specify a bulk density range of 80-120 kg/m\(^{3}\) noting
that it is impossible to give an exact value of density of the sample
under test. The thermal conductivity of dry sample with fibers oriented
parallel to heat flow was found to be 0.0682 W/mK at 20\(^{\circ}\)C.

 Wei et al. (2015) proposed high frequency hot pressing technique
instead of the traditional platen-pressing process for fabrication of
straw based insulation boards. They went on to study the properties of
the straw board manufactured by this technique along with factors
affecting the properties. It is worth noting a fundamental difference in
the structure of such boards; they are composed of straw particles while
conventional straw bales are an aggregate of straw fibers. Also, the
straw boards have a higher density range (180 - 360 kg/m\(^{3}\)).
Steady state measurement of thermal conductivity of such a straw board
with dimensions 300 mm \(\times\) 300 mm \(\times\) 40 mm was conducted
at a mean temperature of 20\(^{\circ}\)C. Over a small range of straw
particle moisture content - 10\% to 18\% - only a slight change in
thermal conductivity - 0.051 W/mK to 0.053 W/mK correspondingly - was
observed. A linear increase in thermal conductivity was reported with
increase in the board density. Moreover, the straw boards exhibited
increased thermal conductivity when particle size in decreased as well
as in conditions of increased ambient temperatures.

 Robinson (2014) cited a previous work, Grmela et al. 2010 which could
not be accessed for the current review, while reporting a thermal
conductivity range from 0.038 W/mK to 0.1 W/mK.

 Lee and Yeom (2014) reported a thermal conductivity value of 0.092 W/mK
for straw of density, 207.5 kg/m\(^{3}\). It is most probably a case a
mistaken use of units for density in this paper. For example, in the
case of straw, it is mentioned to be having a density of 207.5
g/m\(^{3}\) which is unlikely.

 Thermal conductivity measurement of eight straw bale specimens
conducted by Langmans et al. (2014) resulted in values of 0.0589 W/mK
for the case of straw fibers perpendicular to heat flow and 0.0667 W/mK
for the case of straw fibers parallel to heat flow. All specimens have a
density of 88 kg/m\(^{3}\).

Shea et al. (2012) put forward a building system utilizing prefabricated
timber-framed straw based panels. After a good review of previously
reported thermal conductivity values, the authors then tested for the
thermal conductivity of their own samples in a heat flow meter. Six
wheat straw samples with randomly oriented fibers and densities ranging
from 67 kg/m\(^{3}\) to 112 kg/m\(^{3}\) were prepared and placed in a
controlled in a controlled space of 23\(^{\circ}\)C and 50\% RH. At mean
sample temperature of 10\(^{\circ}\), only a modest rise from 0.0594
W/mK to 0.0636 W/mK in thermal conductivity over the corresponding
density range was observed. Further, they reported a 15\% increase in
thermal conductivity when mean sample temperature was increased from
10\(^{\circ}\)C to 30\(^{\circ}\)C.

 Vėjelienė (2012) studied the effect of structural variation on thermal
conductivity of straw, which includes orientation: vertical or
horizontal and mode of processing of straw: chopped straw, prepared from
rotary milling and defibered straw. Further, the work involved use of
additives to straw sample such as graphite particles for improving
infrared radiation absorption and flax oil for better binding. The
measurement of thermal conductivity was carried out in a guarded hot
plate apparatus at a mean sample temperature of 10\(^{\circ}\)C and
under constant ambient conditions of 23\(\pm\)2\(^{\circ}\)C and
50\(\pm\)5 \% RH. Over a density range between 50 kg/m\(^{3}\) and 120
kg/m\(^{3}\), the author developed correlations for each of the
structurally varying straw samples. It was observed that the horizontal
oriented straw showed greater variation with density compared to the
remaining samples types. The magnitudes of thermal conductivity for the
different samples were found to be in the following order: Defibered
straw \textless{} Chopped straw \textless{} Horizontal oriented straw
\textless{} Vertical oriented straw\\

 Ashour et al. (2010), citing his thesis, presented thermal conductivity
values ranging from 0.0414 W/mK to 0.0486 W/mK for Wheat straw and the
same ranging from 0.0353 W/mK to 0.0539 W/mK for Barley straw over
different temperatures. The author noted a greater increase in thermal
conductivity when temperature was raised from 20.7\(^{\circ}\)C to
34.2\(^{\circ}\)C than when temperature was raised from
10.3\(^{\circ}\)C to 20.7\(^{\circ}\)C.\\

 In referring to previous works, Pruteanu (2010) listed the thermal
conductivity results of by McCabe in 1993 and Eisenberg of Oak Ridge
National Laboratory (ORNL) in 1998. The former had reported values
ranging from 0.0481 W/mK to 0.0578 W/mK for straw bales of density 133
kg/m\(^{3}\) while the latter had reported thermal resistance ranging
from 6.5175 W/m\(^{2}\)K to 7.821 W/m\(^{2}\)K for a 55 cm thick straw
bale which translated to a thermal conductivity range of 0.0703 to
0.0844 W/mK. The author then devised a simple technique to measure the
thermal conductivity of straw samples. The first sample having a
moisture content of around 6 \% and density of 51.06 kg/m\(^{3}\) had a
median thermal conductivity value of 0.0614 W/mK and a second sample of
8.25 \% moisture content and 76.40 kg/m\(^{3}\) density had a median
thermal conductivity value of 0.053 W/mK. The median values were
calculated from ten trials on each sample.

 Other reported results not directly accessed: Beck et al. (2003); DIB;
Grelat; Anderson;

 We now look in to some indirect methods used by researchers and the
corresponding results of thermal conductivity obtained.

 On the basis previous literature, Chaussinand et al. (2015) provides a
range of thermal conductivity values, as given below, which is exhibited
for straw owing to its various influencing factors such as composition,
density, moisture content, etc.\\

\[0.052 W/mK \leq \lambda \leq 0.12 W/mK\]

Further, the authors have taken up an existing straw bale building named
ECO46 located in Switzerland for indirect calculation of thermal
properties of straw. Using the fuel consumption data of the building
heating system as well as the on-site weather data in a building
simulation tool, they arrived at a thermal conductivity value of 0.08
\(\pm\) 0.005 W/mK for the straw used in the construction of the
building.\\

\hypertarget{header-n59}{%
\subparagraph{Thermal Diffusivity}\label{header-n59}}

\hypertarget{header-n60}{%
\paragraph{Hygroscopic Properties}\label{header-n60}}

 Hygroscopic properties define the moisture buffering behavior of Straw.

\hypertarget{header-n65}{%
\subsubsection{4. Measurement of Thermo-physical and Hygroscopic
Properties}\label{header-n65}}

\hypertarget{header-n66}{%
\subsubsection{5. Building Simulation techniques for Thermal Comfort
Evaluation}\label{header-n66}}

\hypertarget{header-n67}{%
\subsubsection{6. Thermal and Hygroscopic Performance of Straw based
Buildings through Experiments}\label{header-n67}}

\hypertarget{header-n68}{%
\subsubsection{7. Summary of the Literature Review}\label{header-n68}}

\hypertarget{header-n69}{%
\subsection{METHODOLOGY}\label{header-n69}}

\hypertarget{header-n70}{%
\subsection{REFERENCES}\label{header-n70}}

\end{document}
