\documentclass[12pt]{report}


%% PACKAGES %%


\usepackage{amsmath}

\DeclareMathOperator{\erfc}{erfc}

\usepackage{graphicx}
\usepackage{geometry}
\usepackage{subcaption}
\usepackage{verbatim}
\usepackage{indentfirst} 


%% Global Properties %%

\setlength{\parindent}{3.8em}

%% Document Start %%

\begin{document}

%\title{How to Structure a LaTeX Document}
%\author{Andrew Roberts}
%\date{December 2004}

\chapter*{Experimental Theory and Design}

\section*{Introduction}


There are several steady-state and transient techniques available for the measurement of thermal properties. Although not as accurate as steady-state methods such as the Guarded Hot Plate apparatus, transient techniques offer important advantages over steady-state techniques. These include simultaneous measurement of thermal conductivity and thermal diffusivity, significantly shorter duration of experiments and capability of studying moisture effects on thermal properties.  Moreover, transient measurement setups are generally simpler and could be built at a lower cost compared to the steady-state counterparts.

The transient measurement methods [1]are distinguished by the type (wire, strip or plane) and thermal profile (pulse or step) of heat source and the point of temperature measurement (at source or away from the source). One of the first techniques was implemented by Gustafsson [2] in 1991 using a Transient Plane Source element (now known as the "Gustafsson Probe") which works as both the heat source and the sensor. Transient method involving step-wise heating was discussed by Kubicar and Bohac [3].  Lei et al. [4] used the transient plane source method with stepwise heating to measure thermal properties of fabrics which are fibrous in nature. Recently, Malinarič [5] carried out a detailed analysis of Step-wise transient method with disk shaped plane heat source. 

\section*{Measurement Theory}

In the current work, Transient Plane Source (TPS) with a step heat flux is used for the measurement of thermal conductivity and thermal diffusivity. The temperature profile is captured by a sensor positioned at a fixed distance away from the heat source. 

\subsection*{Semi-infinite Model}

The experimental setup is designed to replicate the 1-D semi-infinite transient model illustrated in Figure 1 during the duration of the experiment. The analytical solution for the model is given by,

\begin{equation}
T(x,t) = \frac{2q}{k}\left[ \sqrt{\frac{\alpha t}{\pi}}\exp{\frac{-x^{2}}{4 \alpha t}} - \frac{x}{2}\erfc{(\frac{x}{2 \sqrt{\alpha t}})}\right]
\end{equation}

where q is the heat flux and x is the distance between the sensor and the heat source. The sample can be approximated as a semi-infinite solid to a certain degree of accuracy if the following criterion is satisfied:
\begin{equation}
Fourier Number  \phantom{1} Fo = \frac{\alpha t }{(L/2)^{2} } \leq 0.2
\end{equation}
where, t is the measurement time and L is the thickness of
the sample. An approximate maximum value of thermal diffusivity,
$\alpha_{max}$ (obtained from literature or other sources) for the
material of the sample or similar material shall be used in the above
condition. By setting a practical value of sample thickness, the maximum
duration of the experiment is obtained as follows.
\begin{equation}
t_{max} = \frac{0.2 (L/2)^{2}}{\alpha_{max}}
\end{equation}
The side length W of the square shaped sample is fixed to be atleast five times
 the distance between the heat source and the temperature sensor as suggested
by Tye et al. {[}{]}.

The temperature profile obtained from the experiment is curve-fit to the
analytical solution to get k and $\alpha$ using non-linear parameter
estimation procedure.


\subsection*{Sensitivity Coefficient Analysis}

The selection of the time window to be used for the curve fitting
procedure depends on the sensitivity of the parameters to the change in
temperature as well as the linear dependence of the parameters to each
other. Sensitivity coefficient analysis is to be carried out for this
purpose.

 Sensitivity Coefficient is defined as
\begin{equation}
\beta_{p}(t) = p \frac{\partial T}{\partial p}
\end{equation}
where p is an unknown parameter. If the sensitivity coefficients are low
or linearly dependent on each other, the curve fit may not work
properly. Here, the unknown parameters are k and $\alpha$ for which
the corresponding sensitivity coefficient expressions are given by,
\begin{equation}
\beta_{k} =\frac{-2q}{k} \left[ \sqrt{\frac{\alpha t}{\pi}}\exp{\frac{-x^{2}}{4 \alpha t}} - \frac{x}{2}\erfc{(\frac{x}{2 \sqrt{\alpha t}})}\right]
\end{equation}

\begin{equation}
\beta_{\alpha} = \frac{q}{k}\sqrt{\frac{\alpha t}{\pi}}\exp{\frac{-x^2}{4 \alpha t}}
\end{equation}
The plot of $\beta_{k}$ and $\beta_{\alpha}$ against time or Fo
indicates the initial period of time during which the properties are not
sensitive to the change in temperature because of which their values
cannot be estimated accurately. Since the less diffusive materials (low
\(\alpha\)) tend to take longer to respond to the applied heat flux, we
shall consider an approximate value of $\alpha_{min}$ of the sample
under study. To apply the sensitivity analysis for different materials, we make use of the
normalized forms of the sensitivity coefficients defined as,
\begin{equation}
\beta_p^{'} =\left|\frac{\beta_p - \beta_{p,0}}{\beta_{p,max} - \beta_{p,0}}\right|
\end{equation}
 where $\beta_{p,0}$ is the sensitivity coefficient value at the start
of the measurement (at time $t = 0$) and and $ \beta_{p,max}$ is the
same at the end of the measurement (at time $t=t_m$ )



A set of parameters are said to be linearly dependent if they satisfy the condition,
%\begin{equation}

%\end{equation}

For the unknown parameters k and $\alpha$, we can write
\begin{equation*}
 C_1 \frac{\partial T}{\partial k} + C_2 \frac{\partial T}{\partial \alpha} = 0 
\end{equation*}


By replacing the partial derivate terms with their non-dimensional equivalents,
\begin{equation*}
 C_1 \beta_{k}^{'} + C_2 \beta_{\alpha}^{'} = 0
\end{equation*}

\begin{equation*}
 \gamma^{'} (say) = \frac{\beta_k^{'}}{\beta_{\alpha}^{'}} = C
\end{equation*}
$C_1$, $C_2$ and $C$ are all constants. Visual inspection of the
 plot of $\gamma^{'}$ against the measurement time gives us some idea 
about the nature of the linear dependence between the two parameters.   

\subsection*{Standard and Difference Analysis}

The standard analysis is carried out by performing the curve fitting by considering the temperature data for 
the time interval $[t,t_{max}]$ where t is increased from 0 to $t_{max}-1$. In difference analysis, we consider the time window $[t - t_s,t]$ where $t_s$ is the chosen time window interval and t is again increased from $t_s$ to $t_{max}$.  The corresponding data is used in the curve fit procedure to obtain k and $\alpha$. By selecting different values of $t_s$, we can arrive at the desired time window in which the influence of the linear dependence of the parameters is nullified to considerable extent. 

Before the start of transient experiments, the standard and difference analyses are carried out using the temperature profile obtained from the theoretical solution (eq. 1) for each sample from which the time window is chosen. However in the actual experiments, random error in the form of noise is inevitably introduced in to the temperature data, the magnitude of which cannot be quantised accurately. Therefore, in addition to performing standard and difference analyses on the ideal temperature profile, we also consider theoretical temperature data in which random error of certain degree is introduced manually. The time window thus obtained from the error-ridden case will be greater than that obtained for the ideal case and more appropriate when planning the experiments. 

\subsection*{Uncertainty Analysis}


For calculating the standard deviation of the estimated parameters, the
uncertainty analysis method described in Beck and Arnold {[}7{]} and
implemented in {[}5{]} is followed here.

 For the set of parameters $p_j$, the standard deviation of the
least-square estimate is obtained as shown below:
\begin{equation}
\sigma^2(p_j) = \{(X^T.X)^{-1}\}_{jj}\sigma^2
\end{equation}
where \(\sigma\) is the standard deviation of the measurement system and
X is the sensitivity matrix defined as,
\begin{equation}
\{X\}_{jj} = \frac{\partial T(t_i, \bar{p})}{\partial p_j}
\end{equation}


\section*{Experimental Setup}

The heating element consisting of a Nickel foil (25 micron thickness),
having a square shape and spiral structure with thin layers of kapton on
either sides acts as the plane heat source. It is heated by a DC power
source supplying constant current. Two identical samples are placed
sandwiching the heat source and held under force through a pressing
mechanism to ensure good contact between the sample and heating element.
Heat conducting paste is also applied on the element to minimize
contact resistance existing because of the presence of air gaps.

Temperature is measured using bare wire K-type thermocouples with bead diameter of about 0.5 mm positioned
by appropriate methods. For the case of solid samples, a hole is drilled
in the radial direction in such a way that the thermocouple is held
tightly in the desired position. In the case of fibrous materials, the
positioning is more difficult and hence a support made of a similar
material to the sample - a thin cylindrical wooden stick with a groove to hold the thermocouple wire - is used to position the thermocouple. Once again, heat conducting paste is used to reduce the
contact resistance existing between the tip of the sensor and the
contact area in the sample. For all the samples considered for study here, the thermocouples
are positioned 1 cm away from the source on both sides.

The entire setup is placed inside a climate
chamber in order to study the effects of temperature and relative
humidity on the thermal properties of the sample. Two HTU21D temperature-humidity sensors
interfaced with Arduino controller are used to monitor the chamber
humidity and temperature. Additionally, two thermocouples also
monitor the chamber temperature. The temperature data is logged in a
desktop computer through the Agilent data acquisition system.
\\

\subsection*{Calibration of Equipment and Instrumentation}
Calibration of the thermocouples and the relative humidity sensors were
carried out before the start of the experiment.




\section*{Results and Discussion}

Polymethyl Methacrylate (PMMA) and Extruded Polystyrene (XPS) are the
reference samples considered for validating the experimental method
while the material under study is rice paddy straw.  The
sample size is determined from the semi-infinite condition mentioned
above. The time window (\(t_0\) to \(t_m\)) for curve-fitting is chosen
by performing the analyses discussed above for each sample. The results
of the analyses are presented in Figure 2 and 3. The final experimental
design values are listed in Table 1.

\begin{table}


\begin{tabular}{|l|c|c|c|}

\hline
& PMMA & XPS & Straw  \tabularnewline

\hline

L (set) {[}\(m\){]} & \(0.03\) & \(0.081\) & \(0.1\)\tabularnewline

\hline

W {[}\(m\){]} & \(0.10\) & \(0.15\) & \(0.18\)\tabularnewline

\hline

\(\rho\) (measured) {[}\(kg/m^3\){]} & \(1172.97\) & \(39.51\) &
\(70-130\)\tabularnewline

\hline

\(k_{max} [W/mK]\) & \(0.21\){[}Rides{]} & \(0.042\){[}Dubois{]} &
\(0.12\) {[}Chaussinand{]}\tabularnewline

\hline


\(C_{min}\){[}\(kJ/kgK\){]} & - & \(1280\){[}Al-Ajlan{]} &
\(1338\){[}Chaussinand{]}\tabularnewline

\hline


\(\alpha_{max}\) {[}\(m^2/s\){]} & \(1.15 \times 10^{-7}\){[}Rides{]} &
\(8.31 \times 10^{-7}\) & \(1.28 \times 10^{-6}\)\tabularnewline

\hline






\(k_{min} [W/mK]\) & \(0.18\){[}Rides{]} & \(0.032\){[}Dubois{]} &
\(0.052\){[}Chaussinand{]}\tabularnewline

\hline


\(C_{max}\){[}\(kJ/kgK\){]} & - & \(1280\) &
\(2000\){[}Chaussinand{]}\tabularnewline

\hline


\(\alpha_{min}\) {[}\(m^2/s\){]} & \(9.0 \times 10^{-8}\){[}Rides{]} &
\(6.33 \times 10^{-7}\) & \(2.0 \times 10^{-7}\)\tabularnewline





\hline


\end{tabular}
\caption[Table caption text]{Parameters considered for measurement analysis for the three different samples} 

\end{table}


\begin{comment}

\begin{table}


\begin{tabular}{|l|c|c|c|}

\hline
& PMMA & XPS & Straw  \tabularnewline

\hline

\(\approx t_0\) {[}\(s\){]} & \(100\) & \(50\) & \(80\)\tabularnewline

\hline

\(\approx t_{max}\){[}\(s\){]} & \(390\) & \(395\) &
\(335\)\tabularnewline

\hline

\end{tabular}
\caption[Table caption text]{Time window chosen for parameter estimation for the three different samples} 

\end{table}
\end{comment}

%% PMMA Analysis %%


\begin{figure}
\begin{subfigure}{1\textwidth}
\centering
\includegraphics[width=7cm,height=7cm]{H:/Dropbox/Research/Experiment/Results/PMMA_SensCoeff.png}
\caption{}
\end{subfigure}\\
\begin{subfigure}{.5\textwidth}
\centering
\includegraphics[width=7cm,height=7cm]{H:/Dropbox/Research/Experiment/Results/PMMA_std_diff_NOerror_GREYSCALE.png}
\caption{}
\end{subfigure}\hfill
\begin{subfigure}{.5\textwidth}
\centering
\includegraphics[width=7cm,height=7cm]{H:/Dropbox/Research/Experiment/Results/PMMA_std_diff_NOerror_alpha_GREYSCALE.png}
\caption{}
\end{subfigure}
\caption{Measurement analysis for PMMA: (a) Sensitivity coefficients and their ratio (b) Standard analysis and difference analysis estimating thermal conductivity for three different time windows (c) Standard analysis and difference analysis estimating thermal diffusivity for three different time windows}

\end{figure}


\begin{figure}
\begin{subfigure}{.5\textwidth}
\centering
\includegraphics[width=7cm,height=7cm]{H:/Dropbox/Research/Experiment/Results/PMMA_std_witherror_GREYSCALE.png}
\caption{}
\end{subfigure}\hfill
\begin{subfigure}{.5\textwidth}
\centering
\includegraphics[width=7cm,height=7cm]{H:/Dropbox/Research/Experiment/Results/PMMA_diff_witherror_GREYSCALE.png}
\caption{}
\end{subfigure}\\
\begin{subfigure}{.5\textwidth}
\centering
\includegraphics[width=7cm,height=7cm]{H:/Dropbox/Research/Experiment/Results/PMMA_std_witherror_alpha_GREYSCALE.png}
\caption{}
\end{subfigure}\hfill
\begin{subfigure}{.5\textwidth}
\centering
\includegraphics[width=7cm,height=7cm]{H:/Dropbox/Research/Experiment/Results/PMMA_diff_witherror_alpha_GREYSCALE.png}
\caption{}
\end{subfigure}
\caption{Measurement analysis for PMMA: (a) Standard analysis with and without random error compared for estimating thermal conductivity (b) Difference analysis with and without error compared considering time window of 100 s for estimating thermal conductivity (c) Standard analysis with and without random error compared for estimating thermal diffusivity(d) Difference analysis with and without error compared considering time window of 100 s for estimating thermal diffusivity}

\end{figure}

%% XPS Analysis %%

\begin{figure}
\begin{subfigure}{1\textwidth}
\centering
\includegraphics[width=7cm,height=7cm]{H:/Dropbox/Research/Experiment/Results/XPS_SensCoeff.png}
\caption{}
\end{subfigure}\\
\begin{subfigure}{.5\textwidth}
\centering
\includegraphics[width=7cm,height=7cm]{H:/Dropbox/Research/Experiment/Results/XPS_std_diff_NOerror_GREYSCALE.png}
\caption{}
\end{subfigure}\hfill
\begin{subfigure}{.5\textwidth}
\centering
\includegraphics[width=7cm,height=7cm]{H:/Dropbox/Research/Experiment/Results/XPS_std_diff_NOerror_alpha_GREYSCALE.png}
\caption{}
\end{subfigure}
\caption{Measurement analysis for XPS: (a) Sensitivity coefficients and their ratio (b) Standard analysis and difference analysis estimating thermal conductivity for three different time windows (c) Standard analysis and difference analysis estimating thermal diffusivity for three different time windows}

\end{figure}


\begin{figure}
\begin{subfigure}{.5\textwidth}
\centering
\includegraphics[width=7cm,height=7cm]{H:/Dropbox/Research/Experiment/Results/XPS_std_witherror_GREYSCALE.png}
\caption{}
\end{subfigure}\hfill
\begin{subfigure}{.5\textwidth}
\centering
\includegraphics[width=7cm,height=7cm]{H:/Dropbox/Research/Experiment/Results/XPS_diff_witherror_GREYSCALE.png}
\caption{}
\end{subfigure}\\
\begin{subfigure}{.5\textwidth}
\centering
\includegraphics[width=7cm,height=7cm]{H:/Dropbox/Research/Experiment/Results/XPS_std_witherror_alpha_GREYSCALE.png}
\caption{}
\end{subfigure}\hfill
\begin{subfigure}{.5\textwidth}
\centering
\includegraphics[width=7cm,height=7cm]{H:/Dropbox/Research/Experiment/Results/XPS_diff_witherror_alpha_GREYSCALE.png}
\caption{}
\end{subfigure}
\caption{Measurement analysis for XPS: (a) Standard analysis with and without random error compared for estimating thermal conductivity (b) Difference analysis with and without error compared considering time window of 100 s for estimating thermal conductivity (c) Standard analysis with and without random error compared for estimating thermal diffusivity(d) Difference analysis with and without error compared considering time window of 100 s for estimating thermal diffusivity}

\end{figure}

%%%%%%%%%%%%%%%%%%%%%%%%%%%

\begin{figure}
\begin{subfigure}{.5\textwidth}
\centering
\includegraphics[width=7cm,height=7cm]{H:/Dropbox/Research/Experiment/Results/XPS_SensCoeff.png}
\caption{}
\end{subfigure}\hfill
\begin{subfigure}{.5\textwidth}
\centering
\includegraphics[width=7cm,height=7cm]{H:/Dropbox/Research/Experiment/Results/XPS_std_diff_NOerror_GREYSCALE.png}
\caption{}
\end{subfigure}\\
\begin{subfigure}{.5\textwidth}
\centering
\includegraphics[width=7cm,height=7cm]{H:/Dropbox/Research/Experiment/Results/XPS_std_witherror_GREYSCALE.png}
\caption{}
\end{subfigure}\hfill
\begin{subfigure}{.5\textwidth}
\centering
\includegraphics[width=7cm,height=7cm]{H:/Dropbox/Research/Experiment/Results/XPS_diff_witherror_GREYSCALE.png}
\caption{}
\end{subfigure}
\caption{Measurement analysis for XPS: (a) Sensitivity coefficients and their ratio (b) Standard analysis and difference analysis for three different time windows (c) Standard analysis with and without random error compared (d) Difference analysis with and without error compared considering time window of 100 s}

\end{figure}

%% Straw %%

\begin{figure}
\begin{subfigure}{1\textwidth}
\centering
\includegraphics[width=7cm,height=7cm]{H:/Dropbox/Research/Experiment/Results/Straw_SensCoeff.png}
\caption{}
\end{subfigure}\\
\begin{subfigure}{.5\textwidth}
\centering
\includegraphics[width=7cm,height=7cm]{H:/Dropbox/Research/Experiment/Results/Straw_std_diff_NOerror_GREYSCALE.png}
\caption{}
\end{subfigure}\hfill
\begin{subfigure}{.5\textwidth}
\centering
\includegraphics[width=7cm,height=7cm]{H:/Dropbox/Research/Experiment/Results/Straw_std_diff_NOerror_alpha_GREYSCALE.png}
\caption{}
\end{subfigure}
\caption{Measurement analysis for Straw: (a) Sensitivity coefficients and their ratio (b) Standard analysis and difference analysis estimating thermal conductivity for three different time windows (c) Standard analysis and difference analysis estimating thermal diffusivity for three different time windows}

\end{figure}


\begin{figure}
\begin{subfigure}{.5\textwidth}
\centering
\includegraphics[width=7cm,height=7cm]{H:/Dropbox/Research/Experiment/Results/Straw_std_witherror_GREYSCALE.png}
\caption{}
\end{subfigure}\hfill
\begin{subfigure}{.5\textwidth}
\centering
\includegraphics[width=7cm,height=7cm]{H:/Dropbox/Research/Experiment/Results/Straw_diff_witherror_GREYSCALE.png}
\caption{}
\end{subfigure}\\
\begin{subfigure}{.5\textwidth}
\centering
\includegraphics[width=7cm,height=7cm]{H:/Dropbox/Research/Experiment/Results/Straw_std_witherror_alpha_GREYSCALE.png}
\caption{}
\end{subfigure}\hfill
\begin{subfigure}{.5\textwidth}
\centering
\includegraphics[width=7cm,height=7cm]{H:/Dropbox/Research/Experiment/Results/Straw_diff_witherror_alpha_GREYSCALE.png}
\caption{}
\end{subfigure}
\caption{Measurement analysis for Straw: (a) Standard analysis with and without random error compared for estimating thermal conductivity (b) Difference analysis with and without error compared considering time window of 100 s for estimating thermal conductivity (c) Standard analysis with and without random error compared for estimating thermal diffusivity(d) Difference analysis with and without error compared considering time window of 100 s for estimating thermal diffusivity}

\end{figure}


%%%%%%%%%%%%%%%%%%%%%%%
\begin{figure}
\begin{subfigure}{.5\textwidth}
\centering
\includegraphics[width=7cm,height=7cm]{H:/Dropbox/Research/Experiment/Results/Straw_SensCoeff.png}
\caption{}
\end{subfigure}\hfill
\begin{subfigure}{.5\textwidth}
\centering
\includegraphics[width=7cm,height=7cm]{H:/Dropbox/Research/Experiment/Results/Straw_std_diff_NOerror_GREYSCALE.png}
\caption{}
\end{subfigure}\\
\begin{subfigure}{.5\textwidth}
\centering
\includegraphics[width=7cm,height=7cm]{H:/Dropbox/Research/Experiment/Results/Straw_std_witherror_GREYSCALE.png}
\caption{}
\end{subfigure}\hfill
\begin{subfigure}{.5\textwidth}
\centering
\includegraphics[width=7cm,height=7cm]{H:/Dropbox/Research/Experiment/Results/Straw_diff_witherror_GREYSCALE.png}
\caption{}
\end{subfigure}
\caption{Measurement analysis for Straw: (a) Sensitivity coefficients and their ratio (b) Standard analysis and difference analysis for three different time windows (c) Standard analysis with and without random error compared (d) Difference analysis with and without error compared considering time window of 100 s}

\end{figure}

\begin{comment}
%% Sensitivity Coefficients Plots %%

\begin{figure}
\begin{subfigure}{.3\textwidth}
\centering
\includegraphics[width=5cm,height=5cm]{H:/Dropbox/Research/Experiment/Results/PMMA_SensCoeff.png}
\caption{}
\end{subfigure}\hfill
\begin{subfigure}{.3\textwidth}
\centering
\includegraphics[width=5cm,height=5cm]{H:/Dropbox/Research/Experiment/Results/PMMA_std_diff_NOerror__GREYSCALE.png}
\caption{}
\end{subfigure}\hfill
\begin{subfigure}{.3\textwidth}
\centering
\includegraphics[width=5cm,height=5cm]{H:/Dropbox/Research/Experiment/Results/PMMA_std_witherror_GREYSCALE.png}
\caption{}
\end{subfigure}
\caption{Sensitivity coefficients and correlation function for (a) PMMA (b) XPS and (c) Straw}

\end{figure}

%%%%%%%%%%%%%%%%%%%%%%%%%%%%%%%%%%%%%%%%%%%%%%%%%%%%%


%% Standard and Difference Analyses without error %%
\begin{figure}
\begin{subfigure}{.3\textwidth}
\centering
\includegraphics[width=5cm,height=5cm]{H:/Dropbox/Research/Experiment/Results/PMMA_std_diff_NOerror__GREYSCALE.png}
\caption{}
\end{subfigure}\hfill
\begin{subfigure}{.3\textwidth}
\centering
\includegraphics[width=5cm,height=5cm]{H:/Dropbox/Research/Experiment/Results/XPS_std_diff_NOerror_GREYSCALE.png}
\caption{}
\end{subfigure}\hfill
\begin{subfigure}{.3\textwidth}
\centering
\includegraphics[width=5cm,height=5cm]{H:/Dropbox/Research/Experiment/Results/Straw_std_diff_NOerror_GREYSCALE.png}
\caption{}
\end{subfigure}
\caption{Standard Analysis and Different Analysis (considering three time windows) for (a) PMMA (b) XPS and (c) Straw}
\end{figure}



%%%%%%%%%%%%%%%%%%%%%%%%%%%%%%%%%%%%%%%%%

%% Standard Analysis with error %%

\begin{figure}
\begin{subfigure}{.3\textwidth}
\centering
\includegraphics[width=5cm,height=5cm]{H:/Dropbox/Research/Experiment/Results/PMMA_std_witherror_GREYSCALE.png}
\caption{}
\end{subfigure}\hfill
\begin{subfigure}{.3\textwidth}
\centering
\includegraphics[width=5cm,height=5cm]{H:/Dropbox/Research/Experiment/Results/XPS_std_witherror_GREYSCALE.png}
\caption{}
\end{subfigure}\hfill
\begin{subfigure}{.3\textwidth}
\centering
\includegraphics[width=5cm,height=5cm]{H:/Dropbox/Research/Experiment/Results/Straw_std_witherror_GREYSCALE.png}
\caption{}
\end{subfigure}
\caption{Standard Analysis with and without random errors compared for (a) PMMA (b) XPS and (c) Straw}
\end{figure}


%%%%%%%%%%%%%%%%%%%%%%%%%%%%%%%%%%%%%%%%%

%% Difference Analysis with error %%

\begin{figure}
\begin{subfigure}{.3\textwidth}
\centering
\includegraphics[width=5cm,height=5cm]{H:/Dropbox/Research/Experiment/Results/PMMA_diff_witherror_GREYSCALE.png}
\caption{}
\end{subfigure}\hfill
\begin{subfigure}{.3\textwidth}
\centering
\includegraphics[width=5cm,height=5cm]{H:/Dropbox/Research/Experiment/Results/XPS_diff_witherror_GREYSCALE.png}
\caption{}
\end{subfigure}\hfill
\begin{subfigure}{.3\textwidth}
\centering
\includegraphics[width=5cm,height=5cm]{H:/Dropbox/Research/Experiment/Results/Straw_diff_witherror_GREYSCALE.png}
\caption{}
\end{subfigure}
\caption{Difference Analysis with and without random errors compared for (a) PMMA (b) XPS and (c) Straw}
\end{figure}
\end{comment}

\end{document}
