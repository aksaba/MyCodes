\documentclass[12pt]{report}


%% PACKAGES %%


\usepackage{amsmath}

\DeclareMathOperator{\erfc}{erfc}

\usepackage{graphicx}
\usepackage{geometry}
\usepackage{subcaption}
\usepackage{verbatim}
\usepackage{indentfirst} 
\usepackage{float}
\usepackage{gensymb}

%% Global Properties %%

\setlength{\parindent}{3.8em}

%% Document Start %%

\begin{document}

%\title{How to Structure a LaTeX Document}
%\author{Andrew Roberts}
%\date{December 2004}

%\chapter*{Experimental Theory and Design}

\section*{Introduction}

\subsection*{Review of previous results of Straw Thermal Properties}

Thermal conductivity for insulations used as construction material for
buildings indicates the capability of the material to prevent heat from
entering/leaving the indoor space which in turn affects the thermal
comfort conditions inside. There exists several methods for evaluation
of thermal conductivity for different types of materials which shall be
briefly touched upon later. For the case of straw, which can be
considered as a fibrous porous material in bulk quantities, an effective
thermal conductivity can be measured through these methods. Straw being
organic and hygroscopic in nature, its effective thermal conductivity is
influenced by an array of factors including type of crop, local climatic
conditions, composition, bulk density, moisture content, etc.

Lebed and Augaitis (2017) measured the thermal conductivity of
triticale straw samples with varying densities in the range of 80
kg/m\(^{3}\) and 190 kg/m\( ^{3}\). Guarded hot plate apparatus was used
to measure the thermal conductivity and the tests were conducted at a
mean sample temperature of 10 \(^{\circ}\)C and ambient conditions of
23\(\pm\)2 \(^{\circ}\)C; 50\(\pm\)5 \% RH. Based on the experimental
data, they had arrived at an equation relating thermal conductivity of
straw and its density as shown below:

\[\lambda_{10^{\circ}C} = -0.00155 + 0.000357.\rho  + \frac{3.381}{\rho}\]

where, \(\rho\) is the density of straw infill. The researchers observed
an increasing trend in thermal conductivity values when density in
increased beyond 120 kg/m\(^{3}\).

 Costes et al. (2017) begin with a good review of previous measurements
of thermal conductivity of straw including its variation with density
and orientation. They argue that samples prepared from straw bales may
not be representative of the actual material due to changes in
orientation and density introduced during the sample preparation process
and consequently influencing its thermal conductivity value. Therefore,
the authors have designed and developed a Guarded hot plate apparatus
capable of measuring thermal conductivity of samples of up to 50 cm
thick. The authors prepared 13 straw bale samples of wide varying
densities (68.1 kg\(/m^{3}\)- 122.7 kg/m\(^{3}\)) but small variations
in thickness (0.3 m - 0.495 m) and measurements were carried out on each
sample five times . Regression of the measurement data resulted in the
following correlation:

\[\lambda = 4.81 \times 10^{-2} + 2.9 \times 10^{-4} d   -1.13 \times 10^{-2} t\]

where d is the density in kg/m\(^{3}\) and t is the thickness in m.

 F. D' Alessandro et al. (2017) also used guarded hot plate apparatus to
measure the thermal conductivity of straw samples of density 80
kg/m\(^{3}\). Two tests conducted resulted in thermal conductivity
values of 0.051 W/mK and 0.053 W/mK. The author notes that the straw
bales were in flat position meaning the heat flow perpendicular to the
straw fiber orientation.

 Douzane et al. (2016) reported thermal conductivity values of straw of
density 80 kg/m\(^{3}\) considering different orientations. Guarded hot
plate techniques was used to to conduct the measurement. At 10°C and dry
conidions, thermal conductivity values of 0.0723 \(\pm\) 0.0014 W/mK was
reported for parallel orientation while 0.0510 \(\pm\) 0.0010 W/mK was
reported for perpendicular orientation. Also measurements performed at
various sample mean temperatures revealed a direct linear relation
between thermal conductivity and sample mean temperature. 

 Conti et al. (2016) attempted in building a low-cost system with
established standards acting as guidelines to measure thermal
conductivity of straw bale with reasonable accuracy. Local agricultural
practices and climatic conditions leading to wide variations in the
thermal behaviour of straw bale is put forward as the reason behind
devising such a system. Thermal conductivities of two straw bale samples
of approximate densities of 65 and 85 kg/m\(^{3}\) were measured to be
0.062 W/mK and 0.070 W/mK respectively.

 Walker et al. (2016) reviewed the previous works in which thermal
conductivity of straw had been measured. It included measurement of
thermal conductivity of 120 kg/m\(^{3}\) dense straw bale by Shea et al.
(2013) who had reported a value of 0.064 W/mK. With regards to
orientation of straw sample, the authors cited the results published by
the German national organisation of straw bale building (FASBA, 2009)
which were equal to 0.067 W/mK for straw oriented parallel to heat flow
and 0.043 W/mK for straw oriented perpendicular to heat flow. Further,
considering straw moisture content as a parameter, the authors reported
the work carried out by a French company CEBTP in 2004 in which a modest
increase from 0.064 to 0.072 W/mK was observed when straw relative
humidity was increased from 0\% to 90\%.

 Samuel et al. (2015) used a previously designed guarded hot plate
apparatus by Dubois and Lebeau (2013) to measure their own sample of
straw. They specify a bulk density range of 80-120 kg/m\(^{3}\) noting
that it is impossible to give an exact value of density of the sample
under test. The thermal conductivity of dry sample with fibers oriented
parallel to heat flow was found to be 0.0682 W/mK at 20\(^{\circ}\)C.

 Wei et al. (2015) proposed high frequency hot pressing technique
instead of the traditional platen-pressing process for fabrication of
straw based insulation boards. They went on to study the properties of
the straw board manufactured by this technique along with factors
affecting the properties. It is worth noting a fundamental difference in
the structure of such boards; they are composed of straw particles while
conventional straw bales are an aggregate of straw fibers. Also, the
straw boards have a higher density range (180 - 360 kg/m\(^{3}\)).
Steady state measurement of thermal conductivity of such a straw board
with dimensions 300 mm \(\times\) 300 mm \(\times\) 40 mm was conducted
at a mean temperature of 20\(^{\circ}\)C. Over a small range of straw
particle moisture content - 10\% to 18\% - only a slight change in
thermal conductivity - 0.051 W/mK to 0.053 W/mK correspondingly - was
observed. A linear increase in thermal conductivity was reported with
increase in the board density. Moreover, the straw boards exhibited
increased thermal conductivity when particle size in decreased as well
as in conditions of increased ambient temperatures.

 Robinson (2014) cited a previous work, Grmela et al. 2010 which could
not be accessed for the current review, while reporting a thermal
conductivity range from 0.038 W/mK to 0.1 W/mK.

 Lee and Yeom (2014) reported a thermal conductivity value of 0.092 W/mK
for straw of density, 207.5 kg/m\(^{3}\). It is most probably a case a
mistaken use of units for density in this paper. For example, in the
case of straw, it is mentioned to be having a density of 207.5
g/m\(^{3}\) which is unlikely.

 Thermal conductivity measurement of eight straw bale specimens
conducted by Langmans et al. (2014) resulted in values of 0.0589 W/mK
for the case of straw fibers perpendicular to heat flow and 0.0667 W/mK
for the case of straw fibers parallel to heat flow. All specimens have a
density of 88 kg/m\(^{3}\).

Shea et al. (2012) put forward a building system utilizing prefabricated
timber-framed straw based panels. After a good review of previously
reported thermal conductivity values, the authors then tested for the
thermal conductivity of their own samples in a heat flow meter. Six
wheat straw samples with randomly oriented fibers and densities ranging
from 67 kg/m\(^{3}\) to 112 kg/m\(^{3}\) were prepared and placed in a
controlled in a controlled space of 23\(^{\circ}\)C and 50\% RH. At mean
sample temperature of 10\(^{\circ}\), only a modest rise from 0.0594
W/mK to 0.0636 W/mK in thermal conductivity over the corresponding
density range was observed. Further, they reported a 15\% increase in
thermal conductivity when mean sample temperature was increased from
10\(^{\circ}\)C to 30\(^{\circ}\)C.

 Vėjelienė (2012) studied the effect of structural variation on thermal
conductivity of straw, which includes orientation: vertical or
horizontal and mode of processing of straw: chopped straw, prepared from
rotary milling and defibered straw. Further, the work involved use of
additives to straw sample such as graphite particles for improving
infrared radiation absorption and flax oil for better binding. The
measurement of thermal conductivity was carried out in a guarded hot
plate apparatus at a mean sample temperature of 10\(^{\circ}\)C and
under constant ambient conditions of 23\(\pm\)2\(^{\circ}\)C and
50\(\pm\)5 \% RH. Over a density range between 50 kg/m\(^{3}\) and 120
kg/m\(^{3}\), the author developed correlations for each of the
structurally varying straw samples. It was observed that the horizontal
oriented straw showed greater variation with density compared to the
remaining samples types. The magnitudes of thermal conductivity for the
different samples were found to be in the following order: Defibered
straw \textless{} Chopped straw \textless{} Horizontal oriented straw
\textless{} Vertical oriented straw

 Ashour et al. (2010), citing his thesis, presented thermal conductivity
values ranging from 0.0414 W/mK to 0.0486 W/mK for Wheat straw and the
same ranging from 0.0353 W/mK to 0.0539 W/mK for Barley straw over
different temperatures. The author noted a greater increase in thermal
conductivity when temperature was raised from 20.7\(^{\circ}\)C to
34.2\(^{\circ}\)C than when temperature was raised from
10.3\(^{\circ}\)C to 20.7\(^{\circ}\)C.

 In referring to previous works, Pruteanu (2010) listed the thermal
conductivity results of by McCabe in 1993 and Eisenberg of Oak Ridge
National Laboratory (ORNL) in 1998. The former had reported values
ranging from 0.0481 W/mK to 0.0578 W/mK for straw bales of density 133
kg/m\(^{3}\) while the latter had reported thermal resistance ranging
from 6.5175 W/m\(^{2}\)K to 7.821 W/m\(^{2}\)K for a 55 cm thick straw
bale which translated to a thermal conductivity range of 0.0703 to
0.0844 W/mK. The author then devised a simple technique to measure the
thermal conductivity of straw samples. The first sample having a
moisture content of around 6 \% and density of 51.06 kg/m\(^{3}\) had a
median thermal conductivity value of 0.0614 W/mK and a second sample of
8.25 \% moisture content and 76.40 kg/m\(^{3}\) density had a median
thermal conductivity value of 0.053 W/mK. The median values were
calculated from ten trials on each sample.\\

We now look in to some indirect methods used by researchers and the
corresponding results of thermal conductivity obtained.

 On the basis previous literature, Chaussinand et al. (2015) provides a
range of thermal conductivity values, as given below, which is exhibited
for straw owing to its various influencing factors such as composition,
density, moisture content, etc.

\[0.052 W/mK \leq \lambda \leq 0.12 W/mK\]

Further, the authors have taken up an existing straw bale building named
ECO46 located in Switzerland for indirect calculation of thermal
properties of straw. Using the fuel consumption data of the building
heating system as well as the on-site weather data in a building
simulation tool, they arrived at a thermal conductivity value of 0.08
\(\pm\) 0.005 W/mK for the straw used in the construction of the
building.


\subsection*{Review of Measurement Methods}
There are several steady-state and transient techniques available for the measurement of thermal properties. Although not as accurate as steady-state methods such as the Guarded Hot Plate apparatus, transient techniques offer important advantages over steady-state techniques. These include simultaneous measurement of thermal conductivity and thermal diffusivity, significantly shorter duration of experiments and capability of studying moisture effects on thermal properties.  Moreover, transient measurement setups are generally simpler and could be built at a lower cost compared to the steady-state counterparts.

The transient measurement methods [1]are distinguished by the type (wire, strip or plane) and thermal profile (pulse or step) of heat source and the point of temperature measurement (at source or away from the source). One of the first techniques was implemented by Gustafsson [2] in 1991 using a Transient Plane Source element (now known as the "Gustafsson Probe") which works as both the heat source and the sensor. Transient method involving step-wise heating was discussed by Kubicar and Bohac [3].  Lei et al. [4] used the transient plane source method with stepwise heating to measure thermal properties of fabrics which are fibrous in nature. Recently, Malinarič [5] carried out a detailed analysis of Step-wise transient method with disk shaped plane heat source. 

\section*{Measurement Theory}

In the current work, Transient Plane Source (TPS) with a step heat flux is used for the measurement of thermal conductivity and thermal diffusivity. The temperature profile is captured by a sensor positioned at a fixed distance away from the heat source. 

\subsection*{Semi-infinite Model}

The experimental setup is designed to replicate the 1-D semi-infinite transient model illustrated in Figure \ref{pic:model} during the duration of the experiment. The analytical solution for the model is given by,

\begin{equation}
T(x,t) = \frac{2q}{k}\left[ \sqrt{\frac{\alpha t}{\pi}}\exp{\frac{-x^{2}}{4 \alpha t}} - \frac{x}{2}\erfc{(\frac{x}{2 \sqrt{\alpha t}})}\right]
\end{equation}

where q is the heat flux and x is the distance between the sensor and the heat source. 

\begin{figure}[H]
\centering
\includegraphics[scale=0.8]{H:/Research_temperory/semi-inf_model.png}
\caption{1-D Semi-infinite model based on which the transient measurement experiment is designed}
\label{pic:model}
\end{figure}

The sample can be approximated as a semi-infinite solid to a certain degree of accuracy if the following criterion is satisfied:
\begin{equation}
Fourier Number  \phantom{1} Fo = \frac{\alpha t }{(L/2)^{2} } \leq 0.2
\end{equation}
where, t is the measurement time and L is the thickness of
the sample. An approximate maximum value of thermal diffusivity,
$\alpha_{max}$ (obtained from literature or other sources) for the
material of the sample or similar material shall be used in the above
condition. By setting a practical value of sample thickness, the maximum
duration of the experiment is obtained as follows.
\begin{equation}
t_{max} = \frac{0.2 (L/2)^{2}}{\alpha_{max}}
\label{eq:semiinf}
\end{equation}
The side length W of the square shaped sample is fixed to be atleast five times
 the distance between the heat source and the temperature sensor as suggested
by Tye et al. {[}{]}.

The temperature profile obtained from the experiment is curve-fit to the
analytical solution to get k and $\alpha$ using non-linear parameter
estimation procedure.


\subsection*{Sensitivity Coefficient Analysis}

The selection of the time window to be used for the curve fitting
procedure depends on the sensitivity of the parameters to the change in
temperature as well as the linear dependence of the parameters to each
other. Sensitivity coefficient analysis is to be carried out for this
purpose.

 Sensitivity Coefficient is defined as
\begin{equation}
\beta_{p}(t) = p \frac{\partial T}{\partial p}
\end{equation}
where p is an unknown parameter. If the sensitivity coefficients are low
or linearly dependent on each other, the curve fit may not work
properly. Here, the unknown parameters are k and $\alpha$ for which
the corresponding sensitivity coefficient expressions are given by,
\begin{equation}
\beta_{k} =\frac{-2q}{k} \left[ \sqrt{\frac{\alpha t}{\pi}}\exp{\frac{-x^{2}}{4 \alpha t}} - \frac{x}{2}\erfc{(\frac{x}{2 \sqrt{\alpha t}})}\right]
\end{equation}

\begin{equation}
\beta_{\alpha} = \frac{q}{k}\sqrt{\frac{\alpha t}{\pi}}\exp{\frac{-x^2}{4 \alpha t}}
\end{equation}
The plot of $\beta_{k}$ and $\beta_{\alpha}$ against time or Fo
indicates the initial period of time during which the properties are not
sensitive to the change in temperature because of which their values
cannot be estimated accurately. Since the less diffusive materials (low
\(\alpha\)) tend to take longer to respond to the applied heat flux, we
shall consider an approximate value of $\alpha_{min}$ of the sample
under study. To apply the sensitivity analysis for different materials, we make use of the
normalized forms of the sensitivity coefficients defined as,
\begin{equation}
\beta_p^{'} =\left|\frac{\beta_p - \beta_{p,0}}{\beta_{p,max} - \beta_{p,0}}\right|
\end{equation}
 where $\beta_{p,0}$ is the sensitivity coefficient value at the start
of the measurement (at time $t = 0$) and and $ \beta_{p,max}$ is the
same at the end of the measurement (at time $t=t_max$ )



For a function $ \eta(x,t,p_j) $, the set of parameters $p_j $ are said to be linearly dependent if they satisfy the condition,
\begin{equation}
	C_1 \frac{\partial \eta}{\partial p_1} + C_2 \frac{\partial \eta}{\partial p_2} + ... +  C_j \frac{\partial \eta}{\partial p_j} = 0
\end{equation}

For the current case with unknown parameters k and $\alpha$, we can write
\begin{equation*}
 C_1 \frac{\partial T}{\partial k} + C_2 \frac{\partial T}{\partial \alpha} = 0 
\end{equation*}


By replacing the partial derivate terms with their non-dimensional equivalents,
\begin{equation*}
 C_1 \beta_{k}^{'} + C_2 \beta_{\alpha}^{'} = 0
\end{equation*}

\begin{equation*}
 \gamma^{'} (say) = \frac{\beta_k^{'}}{\beta_{\alpha}^{'}} = C
\end{equation*}
$C_1$, $C_2$ and $C$ are all constants. Visual inspection of the
 plot of $\gamma^{'}$ against the measurement time gives us some idea 
about the nature of the linear dependence between the two parameters.   

\subsection*{Standard and Difference Analysis}

The standard analysis is carried out by performing the curve fitting by considering the temperature data for 
the time interval $[t,t_{max}]$ where t is increased from 0 to $t_{max}-1$. In difference analysis, we consider the time window $[t - t_s,t]$ where $t_s$ is the chosen time window interval and t is again increased from $t_s$ to $t_{max}$.  The corresponding data is used in the curve fit procedure to obtain k and $\alpha$. By selecting different values of $t_s$, we can arrive at the desired time window in which the influence of the linear dependence of the parameters is nullified to considerable extent. 

Before the start of transient experiments, the standard and difference analyses are carried out using the temperature profile obtained from the theoretical solution (eq. 1) for each sample from which the time window is chosen. However in the actual experiments, random error in the form of noise is inevitably introduced in to the temperature data, the magnitude of which cannot be quantised accurately. Therefore, in addition to performing standard and difference analyses on the ideal temperature profile, we also consider theoretical temperature data in which random error of certain degree is introduced manually. The time window thus obtained from the error-ridden case will be greater than that obtained for the ideal case and more appropriate when planning the experiments. 

\subsection*{Uncertainty Analysis}


For calculating the standard deviation of the estimated parameters, the
uncertainty analysis method described in Beck and Arnold {[}7{]} and
subsequently implemented in {[}5{]} is followed here.

 For the set of parameters $p_j$, the standard deviation of the
least-square estimate is obtained as shown below:
\begin{equation}
\sigma^2(p_j) = \{(X^T.X)^{-1}\}_{jj}\sigma^2
\end{equation}
where \(\sigma\) is the standard deviation of the measurement system and
X is the sensitivity matrix defined as,
\begin{equation}
\{X\}_{jj} = \frac{\partial T(t_i, \bar{p})}{\partial p_j}
\end{equation}


\section*{Experimental Setup}

The heating element consisting of a Nickel foil (25 micron thickness),
having a square shape and spiral structure with thin layers of kapton on
either sides acts as the plane heat source. It is heated by a DC power
source supplying constant current. Two identical samples are placed
sandwiching the heat source and held under force through a pressing
mechanism to ensure good contact between the sample and heating element.
Heat conducting paste is also applied on the element to minimize
contact resistance existing because of the presence of air gaps.

Temperature is measured using bare wire K-type thermocouples with bead diameter of about 0.5 mm positioned
by appropriate methods. For the case of solid samples, a hole is drilled
in the radial direction in such a way that the thermocouple is held
tightly in the desired position. In the case of fibrous materials, the
positioning is more difficult and hence a support made of a similar
material to the sample - a thin cylindrical wooden stick with a groove to hold the thermocouple wire - is used to position the thermocouple. Once again, heat conducting paste is used to reduce the
contact resistance existing between the tip of the sensor and the
contact area in the sample. For all the samples considered for study here, the thermocouples
are positioned 1 cm away from the source on both sides.

The entire setup is placed inside a climate
chamber in order to study the effects of temperature and relative
humidity on the thermal properties of the sample. Two HTU21D temperature-humidity sensors
interfaced with Arduino controller are used to monitor the chamber
humidity and temperature. Additionally, two thermocouples also
monitor the chamber temperature. The temperature data is logged in a
desktop computer through the Agilent data acquisition system.
Calibration of the thermocouples and the relative humidity sensors were
carried out before the start of the experiment.




\section*{Results and Discussion}

Polymethyl Methacrylate (PMMA) and Extruded Polystyrene (XPS) are the
reference samples considered for validating the experimental method
while the material under study is rice paddy straw.  The
sample size is determined from the semi-infinite condition mentioned
above. The time window (\(t_0\) to \(t_{max}\)) for curve-fitting is chosen
by performing the analyses discussed above for each sample. The values used for the analyses are listed in Table \ref{table:values}. Using the semi-infinte condition (eq. \ref{eq:semiinf}), $t_{max}$ is calculated to be about 390 to 400 s for all three samples.  The results
of the analyses for the three samples are presented in Figures \ref{pic:pmma1} to \ref{pic:straw2}. The difference analysis is performed for three time interval $t_s$ values: 25 s, 50 s and 100 s.

On comparing the sensitivity coefficient curves (Figures \ref{pic:pmma1}a, \ref{pic:xps1}a and \ref{pic:straw1}a it is clear that a greater portion of initial temperature data needs to be neglected for PMMA (up to 90 s) owing to its lower diffusivity. The data up to about 25 s and 50 s can be neglected for the case of XPS and Straw respectively. Observing the curves of $\gamma^{'}$ for the three samples, XPS exhibits greater non-linearity than Straw and PMMA samples. As a result the correct values of thermal conductivity and thermal diffusivity are predicted much earlier and with a smaller time window ($t_s = 25 s$) in the case of XPS as seen in the figures \ref{pic:xps1}b and \ref{pic:xps1}c.

To study the effects of noise on the curve-fitting procedure, we perform three trials (a,b and c) each for two cases. In the first case, we introduce an average random error of $0.01^{\circ}$C in to the temperature data while in the second case, a random error of $0.03^{\circ}$C is introduced. Standard and Difference analysis (with $t_s$ = 100s) are performed for the error-ridden temperature data and the results are presented in figures \ref{pic:pmma2}, \ref{pic:xps2} and \ref{pic:straw2} for the samples PMMA, XPS and Straw respectively. The curve-fitting is observed to perform much better for XPS and Straw when compared to PMMA sample. As such, we can conclude that it may be necessary to subject PMMA to more number of experimental trials to get an accurate prediction of the thermal properties. 

Therefore the final time windows during which the temperature data is selected for curve-fitting are: 100 to 390 s for PMMA, 30 to 390 s for XPS and 50 to 350 s for straw. Since the maximum thermal diffusivity of straw is variable owing to several factors, the maximum time may need to be modified depending up on initial experiment results. For all samples, the final value of thermal conductivity or diffusivity is obtained by averaging the estimations resulting from curve-fitting the data over a range of time windows (say, from $t_0$-$t_i$ to  $t_0$-$t_{i+x}$ during which the results are consistent).    

\begin{table}[H]


\begin{tabular}{|l|c|c|c|}

\hline
& PMMA & XPS & Straw  \tabularnewline

\hline

L (set) {[}\(m\){]} & \(0.03\) & \(0.081\) & \(0.1\)\tabularnewline

\hline

W {[}\(m\){]} & \(0.10\) & \(0.15\) & \(0.18\)\tabularnewline

\hline

\(\rho\) (measured) {[}\(kg/m^3\){]} & \(1172.97\) & \(39.51\) &
\(70-130\)\tabularnewline

\hline

\(k_{max} [W/mK]\) & \(0.21\){[}Rides{]} & \(0.042\){[}Dubois{]} &
\(0.12\) {[}Chaussinand{]}\tabularnewline

\hline


\(C_{min}\){[}\(kJ/kgK\){]} & - & \(1280\){[}Al-Ajlan{]} &
\(1338\){[}Chaussinand{]}\tabularnewline

\hline


\(\alpha_{max}\) {[}\(m^2/s\){]} & \(1.15 \times 10^{-7}\){[}Rides{]} &
\(8.31 \times 10^{-7}\) & \(1.28 \times 10^{-6}\)\tabularnewline

\hline






\(k_{min} [W/mK]\) & \(0.18\){[}Rides{]} & \(0.032\){[}Dubois{]} &
\(0.052\){[}Chaussinand{]}\tabularnewline

\hline


\(C_{max}\){[}\(kJ/kgK\){]} & - & \(1280\) &
\(2000\){[}Chaussinand{]}\tabularnewline

\hline


\(\alpha_{min}\) {[}\(m^2/s\){]} & \(9.0 \times 10^{-8}\){[}Rides{]} &
\(6.33 \times 10^{-7}\) & \(2.0 \times 10^{-7}\)\tabularnewline





\hline


\end{tabular}
\caption[Table caption text]{Parameters considered for measurement analysis for the three different samples} 
\label{table:values}
\end{table}


\begin{comment}

\begin{table}


\begin{tabular}{|l|c|c|c|}

\hline
& PMMA & XPS & Straw  \tabularnewline

\hline

\(\approx t_0\) {[}\(s\){]} & \(100\) & \(50\) & \(80\)\tabularnewline

\hline

\(\approx t_{max}\){[}\(s\){]} & \(390\) & \(395\) &
\(335\)\tabularnewline

\hline

\end{tabular}
\caption[Table caption text]{Time window chosen for parameter estimation for the three different samples} 

\end{table}
\end{comment}

%% PMMA Analysis %%


\begin{figure}
\begin{subfigure}{1\textwidth}
\centering
\includegraphics[width=7cm,height=7cm]{H:/Dropbox/Research/Experiment/Results/PMMA_SensCoeff.png}
\caption{}
\end{subfigure}\\
\begin{subfigure}{.5\textwidth}
\centering
\includegraphics[width=7cm,height=7cm]{H:/Dropbox/Research/Experiment/Results/PMMA_std_diff_NOerror_GREYSCALE.png}
\caption{}
\end{subfigure}\hfill
\begin{subfigure}{.5\textwidth}
\centering
\includegraphics[width=7cm,height=7cm]{H:/Dropbox/Research/Experiment/Results/PMMA_std_diff_NOerror_alpha_GREYSCALE.png}
\caption{}
\end{subfigure}
\caption{Measurement analysis for PMMA: (a) Sensitivity coefficients and their ratio (b) Standard analysis and difference analysis estimating thermal conductivity for three different time windows (c) Standard analysis and difference analysis estimating thermal diffusivity for three different time windows}
\label{pic:pmma1}
\end{figure}


\begin{figure}
\begin{subfigure}{.5\textwidth}
\centering
\includegraphics[width=7cm,height=7cm]{H:/Dropbox/Research/Experiment/Results/PMMA_std_witherror_GREYSCALE.png}
\caption{}
\end{subfigure}\hfill
\begin{subfigure}{.5\textwidth}
\centering
\includegraphics[width=7cm,height=7cm]{H:/Dropbox/Research/Experiment/Results/PMMA_diff_witherror_GREYSCALE.png}
\caption{}
\end{subfigure}\\
\begin{subfigure}{.5\textwidth}
\centering
\includegraphics[width=7cm,height=7cm]{H:/Dropbox/Research/Experiment/Results/PMMA_std_witherror_alpha_GREYSCALE.png}
\caption{}
\end{subfigure}\hfill
\begin{subfigure}{.5\textwidth}
\centering
\includegraphics[width=7cm,height=7cm]{H:/Dropbox/Research/Experiment/Results/PMMA_diff_witherror_alpha_GREYSCALE.png}
\caption{}
\end{subfigure}
\caption{Measurement analysis for PMMA: (a) Standard analysis with and without random error compared for estimating thermal conductivity (b) Difference analysis with and without error compared considering time window of 100 s for estimating thermal conductivity (c) Standard analysis with and without random error compared for estimating thermal diffusivity(d) Difference analysis with and without error compared considering time window of 100 s for estimating thermal diffusivity}
\label{pic:pmma2}
\end{figure}

%% XPS Analysis %%

\begin{figure}
\begin{subfigure}{1\textwidth}
\centering
\includegraphics[width=7cm,height=7cm]{H:/Dropbox/Research/Experiment/Results/XPS_SensCoeff.png}
\caption{}
\end{subfigure}\\
\begin{subfigure}{.5\textwidth}
\centering
\includegraphics[width=7cm,height=7cm]{H:/Dropbox/Research/Experiment/Results/XPS_std_diff_NOerror_GREYSCALE.png}
\caption{}
\end{subfigure}\hfill
\begin{subfigure}{.5\textwidth}
\centering
\includegraphics[width=7cm,height=7cm]{H:/Dropbox/Research/Experiment/Results/XPS_std_diff_NOerror_alpha_GREYSCALE.png}
\caption{}
\end{subfigure}
\caption{Measurement analysis for XPS: (a) Sensitivity coefficients and their ratio (b) Standard analysis and difference analysis estimating thermal conductivity for three different time windows (c) Standard analysis and difference analysis estimating thermal diffusivity for three different time windows}
\label{pic:xps1}
\end{figure}


\begin{figure}
\begin{subfigure}{.5\textwidth}
\centering
\includegraphics[width=7cm,height=7cm]{H:/Dropbox/Research/Experiment/Results/XPS_std_witherror_GREYSCALE.png}
\caption{}
\end{subfigure}\hfill
\begin{subfigure}{.5\textwidth}
\centering
\includegraphics[width=7cm,height=7cm]{H:/Dropbox/Research/Experiment/Results/XPS_diff_witherror_GREYSCALE.png}
\caption{}
\end{subfigure}\\
\begin{subfigure}{.5\textwidth}
\centering
\includegraphics[width=7cm,height=7cm]{H:/Dropbox/Research/Experiment/Results/XPS_std_witherror_alpha_GREYSCALE.png}
\caption{}
\end{subfigure}\hfill
\begin{subfigure}{.5\textwidth}
\centering
\includegraphics[width=7cm,height=7cm]{H:/Dropbox/Research/Experiment/Results/XPS_diff_witherror_alpha_GREYSCALE.png}
\caption{}
\end{subfigure}
\caption{Measurement analysis for XPS: (a) Standard analysis with and without random error compared for estimating thermal conductivity (b) Difference analysis with and without error compared considering time window of 100 s for estimating thermal conductivity (c) Standard analysis with and without random error compared for estimating thermal diffusivity(d) Difference analysis with and without error compared considering time window of 100 s for estimating thermal diffusivity}
\label{pic:xps2}
\end{figure}

%%%%%%%%%%%%%%%%%%%%%%%%%%%

%% Straw %%

\begin{figure}
\begin{subfigure}{1\textwidth}
\centering
\includegraphics[width=7cm,height=7cm]{H:/Dropbox/Research/Experiment/Results/Straw_SensCoeff.png}
\caption{}
\end{subfigure}\\
\begin{subfigure}{.5\textwidth}
\centering
\includegraphics[width=7cm,height=7cm]{H:/Dropbox/Research/Experiment/Results/Straw_std_diff_NOerror_GREYSCALE.png}
\caption{}
\end{subfigure}\hfill
\begin{subfigure}{.5\textwidth}
\centering
\includegraphics[width=7cm,height=7cm]{H:/Dropbox/Research/Experiment/Results/Straw_std_diff_NOerror_alpha_GREYSCALE.png}
\caption{}
\end{subfigure}
\caption{Measurement analysis for Straw: (a) Sensitivity coefficients and their ratio (b) Standard analysis and difference analysis estimating thermal conductivity for three different time windows (c) Standard analysis and difference analysis estimating thermal diffusivity for three different time windows}
\label{pic:straw1}
\end{figure}


\begin{figure}
\begin{subfigure}{.5\textwidth}
\centering
\includegraphics[width=7cm,height=7cm]{H:/Dropbox/Research/Experiment/Results/Straw_std_witherror_GREYSCALE.png}
\caption{}
\end{subfigure}\hfill
\begin{subfigure}{.5\textwidth}
\centering
\includegraphics[width=7cm,height=7cm]{H:/Dropbox/Research/Experiment/Results/Straw_diff_witherror_GREYSCALE.png}
\caption{}
\end{subfigure}\\
\begin{subfigure}{.5\textwidth}
\centering
\includegraphics[width=7cm,height=7cm]{H:/Dropbox/Research/Experiment/Results/Straw_std_witherror_alpha_GREYSCALE.png}
\caption{}
\end{subfigure}\hfill
\begin{subfigure}{.5\textwidth}
\centering
\includegraphics[width=7cm,height=7cm]{H:/Dropbox/Research/Experiment/Results/Straw_diff_witherror_alpha_GREYSCALE.png}
\caption{}
\end{subfigure}
\caption{Measurement analysis for Straw: (a) Standard analysis with and without random error compared for estimating thermal conductivity (b) Difference analysis with and without error compared considering time window of 100 s for estimating thermal conductivity (c) Standard analysis with and without random error compared for estimating thermal diffusivity(d) Difference analysis with and without error compared considering time window of 100 s for estimating thermal diffusivity}
\label{pic:straw2}
\end{figure}


%%%%%%%%%%%%%%%%%%%%%%%

\begin{figure}
\centering
\includegraphics[scale=1]{H:/Dropbox/Research/Experiment/Results/PMMA_Compared.png}
\caption{Thermal conductivity of PMMA compared with Rides et al. []}
\label{pic:pmmak}
\end{figure}


\begin{figure}
\centering
\includegraphics[scale=1]{H:/Dropbox/Research/Experiment/Results/PMMA_alpha_Compared.png}
\caption{Thermal diffusivity of PMMA compared with Rides et al. []}
\label{pic:pmmaalpha}
\end{figure}
\end{document}
